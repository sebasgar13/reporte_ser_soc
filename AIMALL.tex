\documentclass[a4paper,10pt]{article}
%\documentclass[a4paper,10pt]{scrartcl}
\usepackage[utf8]{inputenc}
\usepackage[spanish]{babel}
\usepackage{graphicx}
\usepackage{flushend}

\title{Utilización de AIMALL}
\author{Sebastián García Pineda}
\date{}

\pdfinfo{%
  /Title    ()
  /Author   ()
  /Creator  ()
  /Producer ()
  /Subject  ()
  /Keywords ()
}

\begin{document}
\maketitle
El presente documento tratará de explicar de forma brebe y resumida la utilización de AIMALL. \\
\section{Introducción}
AIMALL es un programa que calcula diversas propiedades de sistemas optimizados con metodos no semi-empiricos 
partiendo de archivo wfx, wfn o g09.
AIMALL contiene dos opciones el modo gráfico y el modo de comandos. Siendo el modo gráfico la opción que permite 
la visualización de las estructuras y el modo consola donde se pueden hacer cálculos, mientras que el modo de comandos 
solo permite realizar los cálculos e imprimir la información en la terminal de estos.

\section{Modo gŕafico}
En el modo gráfico es posible la visualización en 2D  siendo de gran ayuda para las primeras incursiones en este programa.


\section{Linea de comandos}
Para el uso de la línea de comandos se utiliza la siguiente línea de comandos: \\

aimqb [opciones] [archivo .wfx/.wfn/.fchk] \\ 

Donde primero se llama a la funcion con aimqb despues se eligen las opciones a utilizar y por ultimo se escribe el
nombre del archivo. \\
Si se quiere abrir solo el cuadro de opciones sin tener que introducir las opciones en la linea en la terminal solo se debe 
escribir aimqb y el nombre del archivo, esto abre el cuadro de comandos pero no el modo gráfico.\\
Ejemplo: \\

aimqb [archivo .wfx/wfn/.fchk]\\   

Para poder observar todas las opciones de comandos desde terminal se introduce -help, con lo que se indicaran las diferentes
opciones así como sus valores predeterminada. \\
Ejemplo:\\

aimqb -help\\

Se utiliza –nogui para que el cuadro de dialogo se observe desde la terminal sin abrir la ventana de registro de AIMQB. 
(aimqb –nogui –help )\\
La opcion de -help arrojara las siguientes líneas de comandos donde se observan las opciones predeterminadas 
(lado derecho para cada opcion), si se quiere dejar las opciones predeterminadas no es necesario escribir todas las lineas 
de comandos en la terminal solo se debera escribir las opciones que se desea cambiar el valor, o si se desea dejar todo 
predeterminado se escribe: \\

aimqb -nogui archivo  \\

Se agregaría –nogui para que la información se imprima en la terminal y no en la 
ventana de registro de AIMQB, esto es importante si se quiete tomar tiempos de calculo. \\


\end{document}
